% Use listings for inclustion of code here I've setup basic shell script
\usepackage{listings} % A rich verbatum environment for code snippets
%\usepackage{fontawesome} % For funky symbols -- didn't use it in the end
\usepackage{multimedia} % Not sure how overleaf deals with this
\usepackage{grffile} % Fix Filename dummies thrown out of prams
\usepackage{tabularx,booktabs} % Use a richer tabular environment
\usepackage{lipsum}% Rhubarb rhubarb rhubarb. Saves having to type rhubard...

\usepackage{hologo}% Provides e.g. \pdfLaTeX
\usepackage{caption}% Better figure captioning

%We have arrows in the commands.tex slide deck
\usepackage{tikz}
\usetikzlibrary{arrows.meta,arrows}

%Adds a part page at \appendex command
\usepackage{appendix}

%References to footnotes
%https://tex.stackexchange.com/questions/16866/how-can-i-use-footnotemark-with-a-ref-argument
\usepackage{refcount}

%Suppress warnings when multiple PDFs with Page Groups are included on a single page
%https://tex.stackexchange.com/questions/76273/multiple-pdfs-with-page-group-included-in-a-single-page-warning#78020
\pdfsuppresswarningpagegroup=1

%Suppress syntax highlighting in comments of listing environments -- a hack
%https://tex.stackexchange.com/questions/378307/advanced-string-highlighting-in-listings#answer-378367
\def\bluecolorifnotalreadypurple{%
    \extractcolorspec{.}\currentcolor
    \extractcolorspec{purple}\stringcolor
    \ifx\currentcolor\stringcolor\else
        \color{blue}%
    \fi
}

%Colourize bash script highlighting in listing environments
\lstset{ 
  language=bash,
  showstringspaces=false,
  commentstyle=\color{red}, 
  keywordstyle=\bluecolorifnotalreadypurple,
  breaklines=true,
  breakatwhitespace=true,
% Can't have this next line in overleaf no colour allowed here!?
%  postbreak=\mbox{\textcolor{red}{$\hookrightarrow$}\space},
  postbreak=\mbox{$\hookrightarrow$\space},
  stringstyle=\color{purple},
  moredelim=[s][\color{purple}]'',
  morestring={**[s][\color{purple}]{<<'EOF'}{EOF}},
  morestring={**[s][\color{purple}]{<<EOF}{EOF}},
  basicstyle=\ttfamily,
  numbers=left,
  numberstyle=\tiny\color{gray}, % because america
  numbersep=5pt
}

% Patch Listing's line number to match lines of included section of file
%https://tex.stackexchange.com/questions/26828/first-line-number-in-lstinputlisting-environment#answer-27240
\makeatletter
\patchcmd{\lst@GLI@}% <command>
  {\def\lst@firstline{#1\relax}}% <search>
  {\def\lst@firstline{#1\relax}\def\lst@firstnumber{#1\relax}}% <replace>
  {\typeout{listings firstnumber=firstline}}% <success>
  {\typeout{listings firstnumber not set}}% <failure>
\makeatother

% In verbatum/lstlisting \end{frame} gets confused so can use \begin/end{slide} to safely encapsulate
% I inadvertently solved this by using lstinputlisting in a later revision, but leaving this in in case.
% https://tex.stackexchange.com/questions/467672/problems-with-fragile-frames-in-beamer#467675
\newenvironment{slide}{\begin{frame}[fragile,environment=slide]}{\end{frame}}

% Start the footnotemarks at 2 for each new page
%\usepackage{perpage}
%\MakePerPage[2]{footnote} % I prefer a dagger.
%This sometimes causes "! LaTeX Error: Counter too large."
%Solution: use a better per page hook
%https://tex.stackexchange.com/questions/306396/paracol-how-to-reset-footnote-counter-every-page#306405
\usepackage{everypage}
\AddEverypageHook{\setcounter{footnote}{1}} % resets footnote counter on every page

% Use Symbols in footnotes
\renewcommand{\thefootnote}{\fnsymbol{footnote}}
