%%%%%%%%%%%%%%%%%%%%%%%%%%%%%%%%%%%%%%%%%%%%%%%%%%%%%%%%%%%%%%%%%%%%%
%% Copyright 2020 Mike Jones, <dr.mike.jones@gmail.com>
%% AKA Grey Wolf <mike.jones@mansouthscouts.org>
%% [23rd Manchester (Birch with Fallowfield)]
%% Scout Membership number: 12114313
%
% This file is part of Grey Wolf's Scouts Beamer Theme.
%
% Grey Wolf's Scouts Beamer Theme is free software: you can redistribute
% it and/or modify it under the terms of the GNU General Public License
% as published by the Free Software Foundation, either version 3 of the
% License, or (at your option) any later version.
%
% Grey Wolf's Scouts Beamer Theme is distributed in the hope that it will 
% be useful, but WITHOUT ANY WARRANTY; without even the implied warranty
% of MERCHANTABILITY or FITNESS FOR A PARTICULAR PURPOSE.  See the GNU
% General Public License for more details.
%
% You should have received a copy of the GNU General Public License
% along with Grey Wolf's Scouts Beamer Theme.  If not, see
% <https://www.gnu.org/licenses/>.
%%%%%%%%%%%%%%%%%%%%%%%%%%%%%%%%%%%%%%%%%%%%%%%%%%%%%%%%%%%%%%%%%%%%%

\changecolours{ScoutPurple}
\section{What Are Prime Numbers?}

\begin{frame}
\frametitle{What Are Prime Numbers?}
\begin{definition}
A \alert{prime number} is a number that has exactly two divisors.
\end{definition}
\end{frame}


\begin{frame}
%\frametitle{What Are Prime Numbers?}
\begin{example}
\begin{itemize}
\item 2 is prime (two divisors: 1 and 2).
  \pause
\item 3 is prime (two divisors: 1 and 3).
  \pause
\item 4 is not prime (\alert{three} divisors: 1, 2, and 4).
\end{itemize}
\end{example}
\end{frame}

\begin{frame}
\frametitle{There Is No Largest Prime Number}
\framesubtitle{The proof uses \textit{reductio ad absurdum}.}
\begin{theorem}
There is no largest prime number.
\end{theorem}
\end{frame}

\begin{frame}
%\frametitle{There Is No Largest Prime Number}
\framesubtitle{The proof uses \textit{reductio ad absurdum}.}
\begin{proof}
\begin{enumerate}
\renewcommand{\theenumi}{\Alph{enumi}}
\item<1-> Suppose $p$ were the largest prime no.
\item<2-> Let $q$ be the product of the first $p$ nos.
\item<3-> Then $q + 1$ is not divisible by any of them.
\item<1-> But $q + 1 > 1$, thus divisible by some prime
number not in the first $p$ nos.\qedhere
\end{enumerate}
\end{proof}
\uncover<4->{The proof used \textit{reductio ad absurdum}.}
\end{frame}

\section{What's Still To Do?}
\subsection{Extra text here}

\begin{frame}
\frametitle{What’s Still To Do?}
\begin{block}{Answered Questions}
How many primes are there?
\end{block}
\begin{block}{Open Questions}
Is every even number the sum of two primes?
\end{block}
\end{frame}

\begin{frame}
\frametitle{What’s Still To Do?}
\begin{columns}
\begin{column}{0.5\textwidth}
\begin{block}{Answered Questions}
How many primes are there?
\end{block}
\end{column}
\column{.5\textwidth}
\begin{block}{Open Questions}
Is every even number the sum of two primes?~\cite{Goldbach1742}
\end{block}
\end{columns}
\end{frame}

\begin{frame}[fragile]
\frametitle{An Algorithm For Finding Prime Numbers.}
\tiny
\begin{verbatim}
int main (void)
{
std::vector<bool> is_prime (100, true);
for (int i = 2; i < 100; i++)
if (is_prime[i])
{
std::cout << i << " ";
for (int j = i; j < 100; is_prime [j] = false, j+=i);
}
return 0;
}
\end{verbatim}
\begin{uncoverenv}<2>
Note the use of \verb|std::|.
\end{uncoverenv}
\end{frame}

\begin{frame}
\frametitle{Further Reading}
\begin{thebibliography}{10}
\bibitem{Goldbach1742}[Goldbach, 1742]
Christian Goldbach.
\newblock A problem we should try to solve before the ISPN '43 deadline,
\newblock \emph{Letter to Leonhard Euler}, 1742.
\end{thebibliography}
\end{frame}
